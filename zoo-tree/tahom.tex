\Note{Various classes are introduced by \cite{godoy:hom}.
}

Introduced by \cite[Definition 4.1]{godoy:hom}, $TA_{\text{hom}}$ is the
class of positive, multi-ply AWEDCs (\autoref{sec:zoo-tree/awedc}) whose
equality constraints are subject to Contained and Same-Stated (and, by
implication, Opaque).

Notably, $TA_{\text{hom}}$ is sufficient to capture the family of languages
which are tree-homomorphic images of positive AWEDCs (\ie $TA_{=}$).
\cite[Proposition 4.6]{godoy:hom}.

Also introduced by \cite[Definition 4.1]{godoy:hom}, $TA_{\neq,\text{hom}}$
is similar to $TA_{\text{hom}}$ with the addition of arbitrary disequality
constraints.

For $TA_{\neq,\text{hom}}$, we have:
\autinfo{
  empty={Yes; pumping construction \cite[Corollary 5.11]{godoy:hom}.},
  intersect={No; see below.},
}

\subsection{Intersection Non-closure of $TA_{\text{hom}}$}

\begin{wrapfigure}{l}{3.75in}\centering\begin{tabular}{cc}
	{$\!\begin{aligned}
	\text{nil} &\xrightarrow{\phantom{1=21}} q_f \\
    \text{cons}(q_e, \text{cons}(q_e,q_l)) &\xrightarrow{1=21} q_f \\
    \text{a} &\xrightarrow{\phantom{1=21}} q_e \\
    \text{f}(q_e,q_e) &\xrightarrow{\phantom{1=21}} q_e \\
    \end{aligned}$}
&
	{$\!\begin{aligned}
    \text{cons}(q_e, \text{nil}) &\xrightarrow{\phantom{1=21}} q_l \\
    \text{cons}(q_e, \text{cons}(q_e,q_l)) &\xrightarrow{1=21} q_l \\
    \text{cons}(q_e, q_l) &\xrightarrow{\phantom{1=21}} q_f \\
    \text{a} &\xrightarrow{\phantom{1=21}} q_e \\
    \text{f}(q_e,q_e) &\xrightarrow{\phantom{1=21}} q_e \\
    \end{aligned}$}
\end{tabular}
\caption{$TA_{\text{hom}}$ to exhibit failure of intersection
closure.}
\label{fig:zoo-tree/tahom:icf}
\end{wrapfigure}

The proof sketch here relies on the all-equal separating example of
\autoref{sec:tree-sepex:allequallist}.  We can exhibit $TA_{\text{hom}}$
automata for the set of lists of even length whose elements are from the
infinite set $\mathcal{T}\set{f/2, a/0}$, and whose consecutive elements are
equal (\eg $[A,A,B,B,\ldots,Y,Y,Z,Z]$), as on the left of
\autoref{fig:zoo-tree/tahom:icf}, and for the related set of lists which
have ``one more'' element on the beginning and end (\eg
$[A,B,B,C,C,\ldots,Y,Y,Z]$), as on the right.  In both cases, $q_f$ is the
sole accepting state of the machine.  Note that the former machine is {\em
deterministic} while the latter is top-down deterministic but only bottom-up
deterministic given one symbol of lookahead.  Their intersection would be
the set of non-empty, even-length, all-equal lists $[A,A,A,\ldots,A]$, which
would require an infinite collection of $TA_{\text{hom}}$ transitions, one
for each length.

\Note{This leaves open the question of the intersection closure of
bidirectionally determinstic $TA_{\text{hom}}$.}
