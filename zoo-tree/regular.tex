Regular tree languages are tree automata without constraints with a finite
$\config$ set, directly analagous to the finite string automata case.  They
are typically presented as single-ply constructions, though this is not
essential -- any multi-ply regular automaton has a weakly equivalent
single-ply automaton, and the reverse direction is clearly trivial.

Given a family of trees accepted by a regular tree automata, the set of
strings formed by labels along paths in this set is a regular finite string
language.  The same is true of accepting runs.  The set of strings formed by
leaves of a regular tree automata (its ``leaf language'') is a context-free
language; dually, every context-free language is the leaf language of some
regular tree automaton.%
%
\footnote{While intersection of two context free language is undecidable
(see \autoref{sec:zoo-str/pda-nd}), the intersection of regular tree languages
is decidable: the leaf language of the intersection of two regular tree
languages is a {\em subset} of the intersection of their leaf languages.}

For bottom-up automata, we have:
\autinfo{
  empty={Linear in the size of the automaton; \cite[Thm 1.7.4]{tata}},
  member={ALOGTIME-complete; \cite[Thm 1.7.2]{tata}},
  univ={PTIME for det. aut., EXPTIME for nondet. \cite[Thm 1.7.7]{tata}},
  equiv={Product of sizes for det. aut., exponential for nondet. \cite[Thm 1.7.8]{tata}},
  compl={Yes, but may require determinization; \cite[Thm 1.3.1]{tata}},
  intersect={Yes, determinization not required; \cite[Thm 1.3.1]{tata}},
  union={Yes; \cite[Thm 1.3.1]{tata}},
  determinize={Yes; \cite[Thm 1.1.9]{tata}},
}
It is additionally possible to determine if an automaton accepts only a
singleton set in PTIME.

Non-deterministic top-down automata are equivalent to bottom-up automata
\cite[Thm 1.6.1]{tata}, but deterministic top-down automata are weaker.
\cite[Prop 1.6.2]{tata}.
