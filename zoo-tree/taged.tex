Tree Automata with Global Equality and Disequality (TAGED) were introduced
by \cite{filiot:sltv} and further results were published in
\cite{filiot:tagc}.  A TAGED automaton is a regular tree automaton (with
state set $\sts$) together with two relations:
\begin{itemize}
  \item $=_A \subset \sts^2$ reflexive and symmetric,
  \item $\neq_A \subseteq \sts^2$ irreflexive and symmetric.
\end{itemize}
%
(Note that these relations need not be total: there may exist two states $p$
and $q$ for which neither $p =_A q$ nor $p \neq_A q$ holds.)
%
A run $r$ of a TAGED automaton is a run of the underlying regular automaton
which additionally satisfies, for all positions $p$ and $p'$ in the run,
\begin{itemize}
  \item $\lab{r|_p} =_A    \lab{r|_{p'}} \Rightarrow \tr{r|_p} = \tr{r|_{p'}}$,
  \item $\lab{r|_p} \neq_A \lab{r|_{p'}} \Rightarrow \tr{r|_p} \neq \tr{r|_{p'}}$.
\end{itemize}

\paragraph{Sub-classes}
\begin{itemize}
%
  \item Polarized forms are defined and denoted $TAGED+$ (for which $\neq_A = \emptyset$) and $TAGED-$ ($=_A = \emptyset$).
%
  \item A ``vertically bounded'' TAGED is an automaton in which
there is a static bound on the number of globally-disequal states on each
root-to-leaf path. 
%
\end{itemize}

Unless otherwise indicated, references in this table are to
\cite{filiot:tagc}.
\autinfo{
  member={NP-complete; Prop 1},
  empty={Positive: EXPTIME-complete (Thm 1),
         Negative: NEXPTIME-complete (Thm 2),
         Vert. Bound: 2NEXPTIME (Thm 4)},
  univ={No; Prop 5},
  union={Yes, polarization-preserving; Prop 2},
  intersect={Yes, polarization-preserving; Prop 2},
  compl={No; Prop 4},
  determinize={No; Prop 3},
}

\subsection{Rigid Tree Automata}
\label{sec:zoo-tree/rta}

Rigid Tree Automata (RTA) are introduced by \cite{jacquemard:rta}.  Like
TAGED, a RTA is defined by a regular tree automaton (with state set $\sts$)
and a distinguished set of configurations, $R \subseteq \sts$, and
imposes the global constraint (Def. 2) that $\forall_{p_1,p_2} . \brak{
r(p_1) = r(p_2) \in R } \Rightarrow t\vert_{p_1} = t\vert_{p_2}$.

\subsubsection{Relation to other classes}

RTAs are \dots
%
\begin{itemize}
%
  \item easily seen to be a sub-class of TAGED+ (\cite[Sec.
3.1]{jacquemard:rta}), with the injection trivial.
%
  \item a super-class of TAGED+, at possible exponential space complexity;
see \cite[Prop. 2]{jacquemard:rta} and \cite{filiot:tagc}).
%
\end{itemize}

\Note{Other classes also discussed in Sec. 3: TAC, DAG automata, TA1M,
ACRV.}

Deterministic RTA are a strict sub-class of RTA (Thm. 4) but strictly
include TA without constraints (Thm. 5).

Unless otherwise indicated, references in this table are to
\cite{jacquemard:rta}.
\autinfo{
  empty={Linear time; Sec. 6.1},
  member={NP-complete; Sec. 6.2},
  univ={No, reduction of PCP; Sec. 6.4},
  finite={PTIME; Sec. 6.6},
%
  compl={No; Sec. 4.2},
  determinize={No; Sec. 5.1},
%
  equiv={No, by implication of universality; Sec. 6.5},
  subset={No, by implication of universality; Sec 6.5},
%
  union={Linear time and space; Sec. 4.1},
  intersect={Exponential time and space; Sec. 4.1},
}
