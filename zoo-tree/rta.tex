Rigid Tree Automata (RTA) are introduced by \cite{jacquemard:rta}.

An RTA has a distinguished set of configurations, $R \subseteq \config$, and
imposes the global constraint (Def. 2) that $\forall_{p_1,p_2} . \brak{
r(p_1) = r(p_2) \in R } \Rightarrow t\vert_{p_1} = t\vert_{p_2}$.  (That is,
for each configuration $r \in R$, all nodes in the run labeled with $r$ must
root equal sub-trees.)


\subsection{Relation to other classes}
RTA are a strict sub-class of TAGED (\autoref{sec:zoo-tree/taged}; Sec. 3.1) but are a
super-class of TAGED+ (at possible exponential complexity; see Prop. 2 and
\cite{filiot:tagc}).

\Note{Other classes also discussed in Sec. 3: TAC, DAG automata, TA1M,
ACRV.}

Deterministic RTA are a strict sub-class of RTA (Thm. 4) but strictly
include TA without constraints (Thm. 5).

\autinfo{
  empty={Linear time; Sec. 6.1},
  member={NP-complete; Sec. 6.2},
  univ={No; Sec. 6.4},
  finite={PTIME; Sec 6.6},
%
  compl={No; Sec. 4.2},
  determinize={No; Sec. 5.1},
%
  equiv={No; Sec. 6.5},
%
  union={Linear time and space; Sec. 4.1},
  intersect={Exponential time and space; Sec. 4.1},
}
