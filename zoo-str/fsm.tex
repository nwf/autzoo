Also called finite-state acceptor, or just state machine.  Everyones'
favorite and simplest automaton family.

Deterministic FSMs are completely characterized by a state set, typically
denoted $\mathcal{Q}$, an initial state $q_0 \in \mathcal{Q}$, and a set of
accepting states, $\mathcal{Q}_f \subseteq \mathcal{Q}$.  (In our notation,
we have $\mathcal{C} = \mathcal{Q}$, $c_0 = q_0$, and $\mathcal{C}_f =
\mathcal{Q}_f$.)  $\delta$ simply selects the new state of the machine.

A deterministic FSM (``DFA'') may be rendered as a directed graph whose
vertices are the elements of $\mathcal{Q}$ and whose edges are labeled with
characters of $\Sigma$.  $\delta$ may be read off from the graph by finding
the input configuration and following the (unique!) edge labeled by the
input character to the output configuration.

Nondeterministic FSMs (``NFA''s) have a similar graphical rendering, save
that some edges may be labeled with the empty string $\epsilon$.  The output
of $\delta$ given $(c_i, s)$ is the set of all nodes reachable from $c_i$ by
following a path composed of any number (possibly zero) $\epsilon$-edges,
exactly one $s$-edge, and then any number of $\epsilon$ edges.%
%
\footnote{Computationally, however, faster algorithms exist that perform
only one of the two $\epsilon$-closure stages suggested here.  Typically,
one will traverse only paths labeled by $s\epsilon^*$, having initialized
with $C_0$ as the $\epsilon$ closure of $Q_0$.} For convenience, one will
often see nondeterministic FSMs with initial state {\em sets}; these are
equivalent to a designated single state and $\epsilon$-edges.

See \href{http://wikipedia.org/Finite-state\_Machine}{Wikipedia's treatment}
or \cite[Ch. 1]{sipser:theorycomp} for thourough introductions.

FSMs' recognition corresponds exactly with
\href{http://wikipedia.org/Regular\_language}{regular languages} and as such
they are closed under all regular operations.

\autinfo{
    empty={Yes; $O(v \log v + e \log e)$
           to test reachability of accepting states. \cite{xxx}},
    member={Yes; $O(n)$ in length of input. \cite{xxx}},
    subset={Yes (by implication);
            complementation avoidable in some cases \cite{abdulla:simantichain}},
%
    compl={Yes; take as accepting states $\sts \setminus \stsf$ \cite{xxx}},
%
    intersect={Yes; cross-product construction},
    union={Yes; cross-product construction for DFA, trivial for NFA},
%
    kstar={Yes; $O\paren{\abs{Q_f} + \abs{Q_i}}$ changes to the machine \cite{xxx}},
    kplus={Yes; $O\paren{\abs{Q_f}}$ changes to the machine \cite{xxx}},
%
    reginter={Yes, by definition},
%
    determinize={Yes; at possible exponential space cost. \cite{xxx}}
}

\subsection{Probabilistic Finite State Automata}

\cite{vidal:pfsm} provides an excellent overview of FSMs weighted with
probabilities.  (Unless otherwise indicated, cross-references in this section
refer to that paper.)  The authors distinguish several classes:
%
\begin{itemize}
%
  \item ``Probabilistic Finite-State Automata'' (PFA, \S2.2)
%
  \item ``Acyclic PFA`` (APFA, \S2.2)
%
  \item ``$\lambda$-PFA'' (\S2.3)
%
  \item ``Deterministic Probabilistic Finite-State Automata'' (DPFA, \S2.4)
%
\end{itemize}
PFA are non-deterministic but do not permit empty-label transitions.
$\lambda$-PFA permit empty-label transitions, easing certain constructions,
but are expressively equivalent to PFA (\S4.3).  DPFA are less expressive
(prop. 10) but more readily analysed.

If we ignore the probabilistic annotations, these automata are equivalent to
\hyperref[sec:zoo-str/fsm]{FSM}s.  Therefore, in the tables below, we mean
to test not just the set of strings but the {\em distribution} of strings.
In the case of membership testing, we instead wish to know the
probability assigned to a string.  For unions, we mean the pointwise-mean
of the two distributions (def. 10), while for intersection we mean {\em
coemmision probability} (the pointwise product of the distributions).

For ($\lambda$-)PFA, we have:
\autinfo{
  member={Yes, $O(n\abs{\sts})$.},
  determinize={No; \S4.1; prop 10.}
}
While for DPFA, we have:
\autinfo{
  member={Yes; $O(n)$ in the length of the input.},
  union={No; \S4.2, prop 9 and appendix A.},
  intersect={Yes; \S5},
  minimize={Yes; \S4.1; see \cite{carrasco:learndetreg} and \cite{mohri:fstlsp}}
}

\Note{These tables are quite sparsely populated; some entries are trivial,
while some I've probably overlooked in the paper.}

\subsection{Probabilistic Finite State Transducers}

\Note{Pull from \cite{pereira:wfst}.}
