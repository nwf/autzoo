\subsection{Ranked Tree Languages}

Trees open some new possibilities and so we shall need some new notation and
a new set of operations.  For an excellent and thourough introduction, see
\cite[Preliminaries]{tata}.  We limit ourselves here to a quick summary.

A {\em ranked} tree language $L$ is a set of finite trees over some
\defn{signature} (also \defn{ranked alphabet}) $\Sigma$, with arity function
$\mbox{ar} : \Sigma \to \mathbb{N}$.  Every node of a tree labeled with
$\sigma \in \Sigma$ has exactly $\mbox{ar}\paren{\sigma}$-many children.%
%
\footnote{More formally, a ranked tree language is a subset of the carrier
of the free algebra over $\paren{\Sigma,\mbox{ar}}$.} We use
$\mathcal{T}(\Sigma,\mbox{ar})$ for such a set of ranked trees; often
$\mbox{ar}$ will be implicit and we will just write $\mathcal{T}(\Sigma)$.
We use the notation $\mathcal{T}(\Sigma \sqcup X)$ to mean the set of trees
whose labels come either from $\Sigma$ (with the appropriate arity) or a
(disjoint) set of ``variables'' $X$, with $\mbox{ar}(x \in X) \defeq 0$.
The notation $\mathcal{T}(\Sigma \times \mathcal{Q})$ will be used for trees
whose labels are pairs of elements from $\Sigma$ and $\mathcal{Q}$; we
define $\mbox{ar}(\sigma \times q \in \Sigma \times \mathcal{Q}) \defeq
\mbox{ar}(\sigma)$.

The set-theoretic operations carry over as might be expected.
